\documentclass[bind,a4paper]{mythesis}
\usepackage{thesis}
\usepackage[full,round]{harvard}
%\usepackage{natbib}%\bibpunct{(}{)}{;}{a}{,}{,}
%\usepackage[utf8]{inputenc}
\usepackage[algochapter,algoruled,linesnumbered,vlined,portugues]{algorithm2e}
\usepackage{graphicx}
\usepackage{latexsym}
\usepackage{amssymb,amsmath,amstext,amscd}
\usepackage{amsthm}
\usepackage{fancyhdr,fancybox}
\usepackage{psfrag}
\usepackage{rotating}
\usepackage[brazilian]{babel}
\usepackage{url}
\usepackage{hyperref}
\usepackage{listings}
\usepackage{epstopdf}
\usepackage{makeidx}
\usepackage{authorindex}


\makeindex

%\input{macros}

%% PDF metadata
\makeatletter
\@ifpackageloaded{hyperref}{%
\hypersetup{%
colorlinks=false,
pdftitle = {Dissertação: Aqui você coloca o título da sua dissertação},
pdfsubject = {Dissertação de Mestrado, Nome do Aluno de Mestrado},
pdfkeywords = {Palavras Chave},
pdfauthor = {\textcopyright\ Nome do Aluno de Mestrado}
}
}{}
\makeatother

%% Define the thesis title and author
\title{Dissertação: Aqui você coloca o título da sua dissertação}
\author{Nome do Aluno de Mestrado}

%% Start the document
\begin{document}

\bibliographystyle{dcu}
\bibliographystyle{kluwer}

%% Define the un-numbered front matter (cover pages, rubrik and table of contents)
\begin{frontmatter}
	\selectlanguage{brazilian}
	%% Title
\titlepage[Universidade Federal de Ouro Preto]{
\begin{flushright}
Dissertação submetida ao\\
Instituto de Ciências Exatas e Biológicas\\
Universidade Federal de Ouro Preto\\ 
para obtenção do título de Mestre em Ciência da Computação
\end{flushright}
}


\dedication{\vspace{4cm}
\begin{flushright}
Dedico este trabalho ...
\end{flushright}
}

%% Abstract in portuguese
\begin{abstract}[\smaller Dissertação: Aqui você coloca o título da sua dissertação\\ \vspace*{1cm} \smaller Resumo]
  %\thispagestyle{empty}

Resumo em português...

\end{abstract}

%% Abstract
\begin{abstract}[\smaller Dissertação: Aqui você coloca o título da sua dissertação em inglês \\ \vspace*{1cm} \smaller Abstract]
  %\thispagestyle{empty}

Abstract in English

\end{abstract}

%% Declaration
\begin{declaration}
Esta dissertação é resultado de meu próprio trabalho, exceto onde referência explícita é feita ao trabalho de outros, e não foi submetida para outra qualificação nesta nem em outra universidade.
  \vspace*{1cm}
  \begin{flushright}
    Aluno de Mestrado 
  \end{flushright}
\end{declaration}

%% Acknowledgements
\begin{acknowledgements}

Agradeço a ...

Agradeço à agencia de fomento.

Muito Obrigado.

\end{acknowledgements}

%% Preface
\begin{preface}
Prefácio

\end{preface}

% ToC
\tableofcontents
\listoffigures
\listoftables
\listofalgorithms

%% Strictly optional!
\frontquote%
  {Alguma citação de alguém.}%
  {Nome do citado}


\end{frontmatter}

%% Start the content body of the thesis
\begin{mainmatter}
  %% Actually, more semantic chapter filenames are better, like "chap-bgtheory.tex"
  \chapter*{Nomenclatura}
\addcontentsline{toc}{chapter}{Nomenclatura}
\markboth{Nomemclature}{Nomenclature}

%% Nomenclature

\begin{tabular*}{20cm}{lp{12cm}}
AE & Algoritmo Evolucionário \\
AG & Algoritmo Genético \\
$\flat$ & Bemol \\
CA & Composição Algorítmica\\
CE & Computação Evolutiva\\
dB & Decibéis \\
Hz & Hertz \\
IA & Inteligência Artificial \\
IHC & Interação Homem-Computador \\
ME & Música Evolutiva \\
MIDI & \textit{Musical Instrument Digital Interface} \\
NSGA-II & \textit{Non-dominated Sorting Genetic Algorithm - II} \\
RNA & Redes Neurais Artificiais \\
SST &  \textit{Sound Synthesis Technique} \\
\# & Sustenido\\
\end{tabular*}

%se tiver mais de uma página usar outra tabela
%\begin{tabular*}{20cm}{lp{12cm}}
%$\emph{fem}$ & Força eletromotriz\\
%\end{tabular*}

  \selectlanguage{brazilian}
 % \input{chapresumo}
  \selectlanguage{brazilian}
  \chapter{Introdução}
\label{chap:Intro}

\section{Introdução}


\section{Justificativa}



\section{Objetivos} 



Esta dissertação resultou nas seguintes publicações:

\begin{itemize}
\item Freitas, A. R. R., Guimarães, F. G. (2011). Originality and Diversity in the Artificial Evolution of Melodies. \textit{Genetic and Evolutionary Computation Conference} (GECCO '11).
\item Freitas, A. R. R., Guimarães, F. G. (2011). Melody Harmonization in Evolutionary Music Using Multiobjective Genetic Algoritms. \textit{8th Sound and Music Computing Conference} (SMC '11).
\end{itemize}


\section{Revisão Bibliográfica}


%Colocar os artigos publicados, se houver algum

\section{Organização do Texto}

  \chapter{Definição do Problema}

That is the real problem
  \chapter{Revisão da Bibliografia}
  \input{chap4}
  \chapter{Procedimentos metodológicos e experimentos}
  \chapter{Discussão dos Resultados e Conclusões}
\end{mainmatter}

%% Produce the appendices
\begin{appendices}
	  \selectlanguage{brazilian}
  %	%% The "\appendix" call has already been made in the declaration
%% of the "appendices" environment (see thesis.tex).
\chapter{Apêndice}

\section{Datasheets dos Circuitos Eletrônicos}

Esta seção traz parte dos datasheets que descreve as principais especificações dos circuitos eletrônicos utilizados no desenvolvimento do \emph{hardware} para auto-calibração e linearização de sensores.

Os datasheets completos podem ser encontrados nos sítios dos fabricantes. 

\subsection{Microcontrolador PIC 18F4550}

Acesse o sítio \texttt{http://www.microchip.com}, para visualizar o \emph{datasheet} completo do microcontrolador PIC 18F4550.


%\begin{figure}[!ht]
%	\centering
%		\includegraphics[width=1.00\textwidth]{figs/PIC18F4550Pag1.jpg}
%	\caption{Datasheet PIC 18F4550 - Página 1}
%	\label{fig:datasheet18F4550Pag1}
%\end{figure}
%
%\begin{figure}[!ht]
%	\centering
%		\includegraphics[width=1.00\textwidth]{figs/PIC18F4550Pag2.jpg}
%	\caption{Datasheet PIC 18F4550 - Página 2}
%	\label{fig:datasheet18F4550Pag2}
%\end{figure}
%
\subsection{Amplificador LM324}

No sítio \texttt{http://www.datasheetcatalog.com/}, pode-se acessar o \emph{datasheet} completo do Amplificador Operacional LM342. 

%\begin{figure}[!ht]
%	\centering
%		\includegraphics[width=1.0\textwidth]{figs/LM324Pag1.jpg}
%	\caption{Datasheet LM324 - Página 1}
%	\label{fig:datasheet324Pag1}
%\end{figure}
%
%\begin{figure}[!ht]
%	\centering
%		\includegraphics[width = 1.0\textwidth]{figs/LM324Pag2.jpg}
%	\caption{Datasheet LM324 - Página 2}
%	\label{fig:datasheet324Pag2}
%\end{figure}


\subsection{Módulo ECIO-40P}

Acesse o sítio \texttt{http://www.matrixmultimedia.com}, para mais informações sobre o módulo ECIO-40P.

%\begin{figure}[!ht]
%	\centering
%		\includegraphics[width=5cm]{figs/ECIO.jpg}
%	\caption{Módulo ECIO - 40P}
%	\label{fig:ecio40P}
%\end{figure}

\end{appendices}

%% Produce the un-numbered back matter (e.g. colophon, 
%% bibliography, tables of figures etc., index...)
\begin{backmatter}
	%\begin{colophon}
%
% This thesis was created in \LaTeXe{} using the ``\texttt{hepthesis}'' class~
% produced by A. Buckley and available at: 
% 
% http://www.ctan.org/tex-archive/macros/latex/contrib/hepthesis/hepthesis.pdf
% 
% The algorithms in this thesis were written using the ``\texttt{algorithm2e}'', a  package for algorithms in
% \LaTeXe{} produced by C. Fiorio and available at: 
% 
% http://www.tug.org/tex-archive/help/Catalogue/entries/algorithm2e.html
%\end{colophon}
%
\bibliography{mythesis}


%% If you have time and interest to generate a (decent) index,
%% then you've clearly spent more time on the write-up than the research :)
\printindex


\end{backmatter}

%% Close
\end{document}
